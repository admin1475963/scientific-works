\documentclass{article}

\usepackage[utf8]{inputenc}
\usepackage[english,russian]{babel}
\usepackage{fontspec}
\usepackage{amsmath}
\usepackage{amssymb}
\usepackage{hyperref}
\usepackage{nccthm}
\usepackage{indentfirst}

\setmainfont{Liberation Serif}

\ProofStyleParameters{\bfseries}{Доказательство}
\newtheorem{definition}{Определение}
\newtheorem{theorem}{Теорема}
\newtheorem{note}{Замечание}
\newtheorem{lemma}{Лемма}

\newcommand{\N}{\mathbb{N}}
\newcommand{\A}{\mathcal{A}}

\title{Автоматы со счётчиком}
\author{Илхомов Мухаммад}
\date{}

\begin{document}

\maketitle

\clearpage

\tableofcontents

\clearpage

\section{Введение}

\section{Основные определение и понятия}

Обозначим множества натуральных чисел через \(\N\). Положим \(\N_0 = N \cup
\{0\}\).

Определения автомата совпадает с определением приведенное в
\cite{litautomatatheory}.

\textbf{Автомат с \(n\) счётчиками} называется автомат и \(n\) счётчиков
соединенные к этом автомату. \textbf{Счётчик} представляет собой устройство
которое хранит в себе неотрицательное целое число.

Автомат с \(n\) счётчиками называется автомат вида
\(\A = (A, Q, B, \phi, \psi, q_0, z_1, \ldots, z_n)\). Здесь
\(z_i, i \in \overline{1, n}\) называется \textit{счётчиком}.
Счётчик можно рассматривать как функцию от времени, которая принимает значения
из \(\N_0\). То есть \(z_i(t) \in \N_0\).

Входной алфавит имеет вид: \(A = \{0, 1\}^n\). Соответственно входной символ
момент времени \(t\) имеет вид \(a(t) = (a_1(t), a_2(t), \ldots, a_n(t))\),
где:
\begin{equation*}
  a_i(t) =
  \begin{cases}
    1, z_i(t) > 0 \\
    0, z_i(t) = 0 \\
  \end{cases}
\end{equation*}
То есть входной символ это информация о том, какие счётчики не равны \(0\) и
какие равны \(0\) в данный момент.

Выходной алфавит имеет вид \(B = \{-1, 0, 1\}^n\). Соответственно выходной символ
момент времени \(t\) имеет вид \(b(t) = (b_1(t), b_2(t), \ldots, b_n(t))\).

Выходной символ определяеть как измениться значения счётчиков. \(1\) означает
``добавить 1'', а \(0\) и \(-1\) соответственно ``не изменить'' и
``отнимать 1''.
Формально значения счётчика в момент времени \(t + 1\) определяется следуещим
образом:
\begin{equation*}
  z_i(t + 1) = z_i(t) + b_i(t), i \in \overline{1, n}
\end{equation*}

Сразу заметим, мы предпологаем что выходная функция \(\psi\) является
``корректно'', то есть счётчики никогда не принимает отрицательные значение.
Формально это можно определить вот так: \(z_i(t) = 0 \Rightarrow b_i(t) \geq 0,
i \in \overline{1, n}\).

Выделим множество \textit{финальных состояний} \(Q_f \subseteq Q\).
Автомат со счётчиками останавливается свою работу если \(q(t) \in Q_f\).

Мы рассмотрим вычислимость частично определенных счётнозначной функции из \(m\)
переменных, то есть функцию, которая имеет вид:
\({f: \N_0^m \rightarrow \N_0 \cup \{\lambda\}}\).
\(\lambda\) означаеть ``неопределённость'', то есть считаем
\(f(x_1, \ldots, x_m) = \lambda\), если \(f\) в точке \((x_1, \ldots, x_m)\)
неопределено.

Рассмотрим автомат с \(n\) счётчиками
\(\A = (A, Q, B, \phi, \psi, q_0. z_1, \ldots, z_n)\), где \(n \geq m\).
Для автомата \(\A\) рассмотрим функцию \(f_{\A}\) которая он \textit{вычисляеть}.
Если автомат начинаеть работу с значениями счётчиков \({z_1(0) = x_1, \ldots,
z_m(0) = x_m}\) и \({z_{m + 1}(0) = 0, \dots, z_{n}(0) = 0}\), и останавливается
с не более одним ненулевым счётчиком, то значения функции \(f_{\A}\) на точке
\((x_1, \dots, x_m)\) равно значению этого счётчика. В противном случае
(если автомат не остановится или остановится с более чем одной ненулевым
счётчиком) функция \(f_{\A}\) не определено на данной точке.

\clearpage

\begin{thebibliography}{}
\bibitem{litautomatatheory} Кудрявцев В.Б., Алешин С.В., Подколзин А.С.
  \textit{Введение в теорию автоматов}. Наука, 1985.
\bibitem{litautomatalanguages} Д. Хопкрофт, Р. Мотвани, Д. Ульман
  \textit{Введение в теорию автоматов, языков и вычислений}. 2-е изд. :
  Пер. с англ. --- Москва, Издательский дом ``Вильямс'', 2002.
\bibitem{litautomatacounters} Кузьмин Е.В., Соколов В.А.
  \textit{Автоматные счетчиковые машины}. Ярославль, 2012.
\end{thebibliography}

\end{document}