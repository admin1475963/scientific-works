\documentclass{article}

\usepackage[utf8]{inputenc} \usepackage[english,russian]{babel}
\usepackage{fontspec}
\usepackage{amsmath}
\usepackage{amssymb}
\usepackage{hyperref}
\usepackage{nccthm}
\usepackage{indentfirst}

\setmainfont{Liberation Serif}

\ProofStyleParameters{\bfseries}{Доказательство}
\newtheorem{definition}{Определение}
\newtheorem{theorem}{Теорема}
\newtheorem{note}{Замечание}
\newtheorem{lemma}{Лемма}

\newcommand{\N}{\mathbb{N}}
\newcommand{\Z}{\mathbb{Z}}
\newcommand{\A}{\mathcal{A}}
\newcommand{\F}{\mathcal{F}}

\title{Автоматы со счётчиком}
\author{Илхомов Мухаммад}
\date{}

\begin{document}

\maketitle

\clearpage

\tableofcontents

\clearpage

\section{Введение}

\section{Основные определение и понятия}

Обозначим множества натуральных чисел через \(\N\). Положим \(\N_0 = N \cup
\{0\}\).

Определения автомата совпадает с определением приведенное в
\cite{litautomatatheory}.

В этой работе определение автомата будет такой:

\begin{definition}
  \label{def:automata}
  Автоматом называется семёрку
  \(\A = (A, Q, Q_f, B, \phi, \bar{\psi}, q_0)\), где:
  \begin{itemize}
    \item \(A=\{0, 1\}^n\) --- входной алфавит
    \item \(Q\) --- множество состояний
    \item \(Q_f\) --- множество финальных состояний
    \item \(B=\{-1, 0, 1\}^n\) --- выходной алфавит
    \item \(\phi: (Q \setminus Q_f) \times A \rightarrow Q\) ---
      функция переходов
    \item \(\bar{\psi}: (Q \setminus Q_f) \times A \rightarrow B\) ---
      функция выхода, которая состоит из \(n\) функции: \linebreak
      \(\bar{\psi} = (\psi_1, \psi_2, \ldots, \psi_n)\).
  \end{itemize}

  При этом есть ограничения для \(\bar{\psi}\):

  \begin{equation}
    \label{eq:constraint}
    \forall i \in \{1, 2, \ldots, n\}\ (x_i = 0 \Rightarrow
    \psi_i(q, x_1, x_2, \ldots, x_i, \ldots, x_n) \geq 0)
  \end{equation}
  Число \(n\) называется арностью автомата \(\A\).
\end{definition}

\begin{definition}
  \label{def:automata_with_counters}
  Упарядоченного множество \((\A, z_1^0, z_2^0, \ldots, z_n^0)\)
  называется автомат с \(n\) счётчиками, где
  \begin{itemize}
  \item \(\A\) --- автомат с арностью \(n\)
  \item \(z_i^0 \in \N_0\) --- начальным значениям \(i\)-го счётчика, где
    \(i \in \{1, 2, \ldots, n\}\)
  \end{itemize}
\end{definition}

\begin{definition}
  Пусть у нас есть автомат с \(n\) счётчиками
  \((\A, z_1^0, z_2^0, \ldots, z_n^0)\).
  Каноническими уравнениями автомата называется систему \ref{eq:canonical}
  \begin{equation}
    \label{eq:canonical}
    \begin{cases}
      q(0) = q_0 \\
      z_1(0) = z_1^0 \\
      z_2(0) = z_2^0 \\
      \ldots \\
      z_n(0) = z_n^0 \\
      q(t + 1) = \phi(q(t), a(t)) \\
      z_1(t + 1) = z_1(t) + \psi_1(q(t), a(t)) \\
      z_2(t + 1) = z_2(t) + \psi_2(q(t), a(t)) \\
      \ldots \\
      z_n(t + 1) = z_n(t) + \psi_n(q(t), a(t))
    \end{cases}
    \end{equation}
    где
    \begin{align*}
      & t \in \N_0 \\
      & a(t) = (\text{sign}(z_1(t)), \text{sign}(z_2(t)), \ldots,
      \text{sign}(z_n(t))) \\
      & \text{sign}(x) =
      \begin{cases}
        -1, &x \in (-\infty; 0) \\
        0, &x = 0 \\
        1, &x \in (0; \infty)
      \end{cases} \\
    \end{align*}
\end{definition}

\begin{definition}
  \label{def:behaviour}
  Пусть у нас есть автомат с \(n\) счётчиками
  \((\A, z_1^0, z_2^0, \ldots, z_n^0)\).
  \begin{itemize}
    \item Если существуеть момент времени \(t'\), такое что \(q(t') \in Q_f\),
      тогда поведением автомата с \(n\) счётчиками называется конечная
      последовательность:
      \begin{align*}
        & (q(0), z_1(0), z_2(0), \ldots, z_n(0)) \\
        & (q(1), z_1(1), z_2(1), \ldots, z_n(1)) \\
        & \ldots, \\
        & (q(t'), z_1(t'), z_2(t'), \ldots, z_n(t'))
      \end{align*}
      В этом случае говорится автомат с \(n\) счётчиками остановится.
    \item Если не существуеть момент времени \(t'\), такое что \(q(t') \in Q_f\),
      тогда поведением автомата с \(n\) счётчиками называется бесконечная
      последовательность:
      \begin{align*}
        & (q(0), z_1(0), z_2(0), \ldots, z_n(0)) \\
        & (q(1), z_1(1), z_2(1), \ldots, z_n(1)) \\
        & \ldots, \\
        & (q(t), z_1(t), z_2(t), \ldots, z_n(t')) \\
        & \ldots
      \end{align*}
      В этом случае говорится автомат с \(n\) счётчиками не остановится.
  \end{itemize}
\end{definition}

Автомат с \(n\) счётчиками можно рассмотреть как автомат соединенные с
\(n\) внешными памятам, каждый который можеть хранить неотрицательное число.
Автомат может видить только то что, каждый счётчик равно нулью или нет.
Каждый момент времени автомат может менять значения счётчика не болле чем на
одну единицу, при этом значения счётчиков всегда дожен остаться неотрицательным.
Ограничение \ref{eq:constraint} как раз таки обеспечиваеть неотрицательность
счётчиков.

Мы рассмотрим вычислимость частично определенных счётнозначной функции из \(m\)
переменных, то есть функции, которые имеет вид:
\({f: \N_0^m \rightarrow \N_0 \cup \{\lambda\}}\).
\(\lambda\) означаеть ``неопределённость'', то есть считаем
\(f(x_1, \ldots, x_m) = \lambda\), если \(f\) в точке \((x_1, \ldots, x_m)\)
неопределено.

\begin{definition}
  Будем говорить что автомат \(\A\) с арностью \(n\) вычисляет функция
  \(m\)-переменных \(f_{\A}: \N^m_0 \rightarrow \N_0\) (где \(m \leq n\)), если:
  \begin{itemize}
    \item \(f_{\A}(x_1, x_2, \ldots, x_m) = y\) и автомат с \(n\) счётчиками
      \((\A, x_1, x_2, \ldots, x_m, 0, \ldots, 0)\) остановится,
      причем в момент остановки \(t'\) все счётчики кроме первой равны \(0\),
      а значения первого счётчика равно \(y\)
    \item \(f_{\A}(x_1, x_2, \ldots, x_m) = \lambda\) и автомат с \(n\)
      счётчиками \((\A, x_1, x_2, \ldots, x_m, 0, \ldots, 0)\) остановится,
      причем в момент остановки \(t'\) существует ненулевой счётчик кроме
      первого счётчика
    \item \(f_{\A}(x_1, x_2, \ldots, x_m) = \lambda\) и автомат с \(n\)
      счётчиками \((\A, x_1, x_2, \ldots, x_m, 0, \ldots, 0)\) не остановится
  \end{itemize}
\end{definition}

Мы рассморим функции только от частично определенных функции одной переменной.
Обозначим через \(J_n\) все частично определенных функции одной переменной,
которые вычислимы через автоматы с \(n\) счётчиками.
Очевидно что для любого \(n\) верно \(J_n \subseteq J_{n + 1}\).

\begin{definition}
  Функцию \(f: \N_0 \rightarrow \N_0\) называется периодичным, если существуеть
  \(T \in \N\) и \(T_0 \in \N_0\), такое что для любого \(x > T_0\) и для любого
  \(k \in \N_0\) верно \(f(x) = f(x + kT)\).
\end{definition}

\begin{definition}
  Функцию \(f: \N_0 \rightarrow \N_0\) называется псевдопериодичным, если
  существуеть \(T_0 \in \N_0\), такое что для любого \(x > T_0\) существует
  \(T \in \N\), такое что для любого \(k \in \N_0\) верно \(f(x) = f(x + kT)\).
\end{definition}

\begin{definition}
  \label{def:manoeuvre}
  Пусть у нас есть автомат с двумя счётчиками.
  Конечным маневрем называется поведение автомата от момента времени \(t_1\)
  до \(t_2\):
  \begin{align*}
    & (q(t_1), z_1(t_1), z_2(t_1)), \\
    & (q(t_1 + 1), z_1(t_1 + 1), z_2(t_1 + 1)), \\
    & \ldots, \\
    & (q(t_2), z_1(t_2), z_2(t_2))
  \end{align*}
  если выполняется один из следущих условий:
  \begin{align}
    \label{eq:manoeuvre}
    \begin{cases}
      z_1(t_1) = 0 \\
      \forall t (t_1 < t \leq t_2 \Rightarrow z_2(t) \neq 0) \\
      z_2(t_2) = 0
    \end{cases}
    \begin{cases}
      z_2(t_1) = 0 \\
      \forall t (t_1 < t \leq t_2 \Rightarrow z_1(t) \neq 0) \\
      z_1(t_2) = 0
    \end{cases}
  \end{align}
\end{definition}

В этой работе основном рассматривается автомат с двумя счётчиками.

\begin{definition}
  \label{def:infinity_manoeuvre}
  Пусть у нас есть автомат с двумя счётчиками.
  Бесконечным маневрем называется поведение автомата от момента времени \(t_1\):
  \begin{align*}
    & (q(t_1), z_1(t_1), z_2(t_1)), \\
    & (q(t_1 + 1), z_1(t_1 + 1), z_2(t_1 + 1)), \\
    & \ldots
  \end{align*}
  если выполняется один из следущих условий:
  \begin{align}
    \label{eq:manoeuvre}
    \begin{cases}
      z_1(t_1) = 0 \\
      \forall t (t_1 < t \Rightarrow z_2(t) \neq 0) \\
    \end{cases}
    \begin{cases}
      z_2(t_1) = 0 \\
      \forall t (t_1 < t \Rightarrow z_1(t) \neq 0) \\
    \end{cases}
  \end{align}
\end{definition}

\begin{definition}
  \label{def:manoeuvre_function}
  Функцию \(f\) называется функцию маневра если она имеет вид:
  \begin{equation}
    \label{eq:manoeuvre_function}
    f(x) =
    \begin{cases}
      C_0, & x = 0 \\
      C_1, & x = 1 \\
      \ldots \\
      C_{x_0}, & x = x_0 \\
      \frac{px + r(x)}{q}, & x > x_0
    \end{cases}
  \end{equation}
  где \(r(x)\) периодическая функция с предпериодом \(x_0\) и периодом \(q\),
  причем \(r(x)\) такая, что \(\frac{px + r(x)}{q}\) является целое число для
  всех \(x > x_0\).
\end{definition}

Обозначим через \(\F\) все функции частично определенных функции одной
переменной.

Множество значений функции \(f \in \F\) обозначим через \(Im(f)\).

\begin{definition}
  \label{def:switcher}
  Пусть \(\varphi \in \F\).
  \(\varphi\) называется переключательной функцией, если
  \(Im(\varphi) < \infty\).
\end{definition}

Обозначим через \(\Phi\) множество всех переключательных функции.

\begin{definition}
  \label{def:switcher_rang}
  Пусть \(\varphi \in \Phi\).
  \(Im(\varphi)\) называется рангом функции \(\varphi\).
\end{definition}

\begin{definition}
  \label{def:function_with_switcher}
  Пару \((f, \varphi)\) называется функцию с переключателем,
  если \(f \in \F, \varphi \in \Phi\).
  \(f\) называется основной функцией.
\end{definition}

Обозначим через \(\Psi\) множество всех функции с переключаетелем.
Определим на множество \(\Psi\) два оператора.

\begin{definition}
  \label{def:generalized_superposition}
  Пусть \(S\) --- оператор над множеством \(\Psi\).
  и пусть \((f, \varphi)\), \((f_1, \varphi_1)\), \((f_2, \varphi_2)\), \ldots,
  \((f_n, \varphi)\) --- функции с переключателями,
  \(E(\varphi) = \{k_1, k_2, \ldots, k_n\}\) и
  \((f', \varphi') = S((f, \varphi), (f_1, \varphi), (f_2, \varphi_2), \ldots
  (f_n, \varphi_n))\)
  Тогда оператор \(S\) называется обобщенной суперпозицей, если
  \begin{align}
    \label{eq:generalized_superposition}
    & f'(x) =
      \begin{cases}
        f_1(f(x)), & \varphi(x) = k_1 \\
        f_2(f(x)), & \varphi(x) = k_2 \\
        \ldots \\
        f_n(f(x)), & \varphi(x) = k_n
      \end{cases} \\
    & \varphi'(x) =
      \begin{cases}
        \varphi_1(f(x)), & \varphi(x) = k_1 \\
        \varphi_2(f(x)), & \varphi(x) = k_2 \\
        \ldots \\
        \varphi_n(f(x)), & \varphi(x) = k_n
      \end{cases}
  \end{align}
\end{definition}

Введем некоторые обозначения:
\begin{itemize}
\item
  \(f^n(x)\) --- это \(n\) раз применения функция \(f\) на \(x\)
\item
  Пусть \((f, \varphi) \in \Psi\).
  \begin{align*}
    & \Delta_0(a) = \{x \in \N_0 | \varphi(x) = a\} \\
    & \Delta_1(a) = \{x \in \Delta_0 | \varphi(f(x)) = a\} \\
    & \ldots \\
    & \Delta_k(a) = \{x \in \Delta_{k - 1} | \varphi(f^k(x)) = a\} \\
    & \ldots
  \end{align*}
\item
  Пусть \((f, \varphi) \in \Psi\).
  \begin{align*}
    & X_0(a) = \N_0 \setminus \Delta_0(a) \\
    & X_1(a) = \Delta_0(a) \setminus \Delta_1(a) \\
    & \ldots \\
    & X_k(a) = \Delta_{k - 1}(a) \setminus \Delta_k(a) \\
    & \ldots
  \end{align*}
\item
  Пусть \((f, \varphi) \in \Psi\).
  \begin{equation*}
    X_{\infty}(a) = \N_0 \setminus \bigcup_{k \geq 0} X_k
  \end{equation*}
\end{itemize}

\begin{definition}
  \label{def:special_recusion}
  Пусть \(R_a\) --- оператор над множеством \(\Psi\).
  и пусть \((f, \varphi)\) --- функция с переключателем,
  \((f', \varphi') = R_a(f, \varphi)\).
  Тогда оператор \(R_a\) называется обобщенной суперпозицей, если
  \begin{align}
    \label{eq:generalized_superposition}
    & f'(x) =
      \begin{cases}
        f(x), & x \in X_0(a) \\
        f(f(x)), & x \in X_1(a) \\
        \ldots \\
        f^{k + 1}(x), & x \in X_k(a) \\
        \ldots
      \end{cases} \\
    & \varphi'(x) =
      \begin{cases}
        \varphi(x), & x \in X_0(a) \\
        \varphi(f(x)), & x \in X_1(a) \\
        \ldots \\
        \varphi(f^k(x)), & x \in X_k(a) \\
        \ldots
      \end{cases}
  \end{align}
\end{definition}

\section{Классические результаты}

\begin{theorem}
  \label{theor:one_counter}
  \(J_1\) состоит из периодических функции и функции следующего вида:
  \begin{equation*}
    f(x) =
    \begin{cases}
      C_0, & x = 0 \\
      C_1, & x = 1 \\
      \ldots \\
      C_{k - 1}, & x = k - 1\\
      x + c, & x \geq k
    \end{cases}
  \end{equation*}
  где \(k, c, C_0, C_1, \ldots, C_{k - 1} \in \N_0\).
\end{theorem}

\begin{theorem}
  \label{theor:three_counter}
  \(J_3\) совпадаеть с классом частично рекурсивных функции от одной переменной.
\end{theorem}

\begin{theorem}
  \label{theor:two_counter_special}
  Пусть у нас есть кодировка \(x \mapsto 2^x\).
  В таком кодировке любая частично рекурсивная функция вычислима на автоматах с
  двумя счётчиками.
\end{theorem}

\clearpage

\begin{thebibliography}{}
\bibitem{litautomatatheory} Кудрявцев В.Б., Алешин С.В., Подколзин А.С.
  \textit{Введение в теорию автоматов}. Наука, 1985.
\bibitem{litautomatalanguages} Д. Хопкрофт, Р. Мотвани, Д. Ульман
  \textit{Введение в теорию автоматов, языков и вычислений}. 2-е изд. :
  Пер. с англ. --- Москва, Издательский дом ``Вильямс'', 2002.
\bibitem{litautomatacounters} Кузьмин Е.В., Соколов В.А.
  \textit{Автоматные счетчиковые машины}. Ярославль, 2012.
\end{thebibliography}

\end{document}